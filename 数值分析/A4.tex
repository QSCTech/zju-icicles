\documentclass[]{article}
\usepackage{multicol}
\usepackage{lmodern}
\usepackage{amssymb,amsmath}
\usepackage{ifxetex,ifluatex}
\usepackage{geometry}
\geometry{left=1cm,right=1cm,top=1cm,bottom=1.5cm}
\usepackage{xeCJK} % 可以使用中文,而且有粗体,斜体
\setCJKmainfont{宋体-简 常规体} % 设置默认中文字体,不影响默认英文字体
\newCJKfontfamily\Hei{冬青黑体简体中文 W3}   %定義指令\Hei則切換成正黑體
\usepackage{fixltx2e} % provides \textsubscript
% \ifnum 0\ifxetex 1\fi\ifluatex 1\fi=0 % if pdftex
%   \usepackage[T1]{fontenc}
%   \usepackage[utf8]{inputenc}
% \else % if luatex or xelatex
%   \ifxetex
%     \usepackage{mathspec}
%   \else
%     \usepackage{fontspec}
%   \fi
%   \defaultfontfeatures{Ligatures=TeX,Scale=MatchLowercase}
% \fi
% use upquote if available, for straight quotes in verbatim environments
\IfFileExists{upquote.sty}{\usepackage{upquote}}{}
% use microtype if available
\IfFileExists{microtype.sty}{%
\usepackage[]{microtype}
\UseMicrotypeSet[protrusion]{basicmath} % disable protrusion for tt fonts
}{}
\PassOptionsToPackage{hyphens}{url} % url is loaded by hyperref
\usepackage[unicode=true]{hyperref}
\hypersetup{
            pdfborder={0 0 0},
            breaklinks=true}
\urlstyle{same}  % don't use monospace font for urls
\usepackage{graphicx,grffile}
\makeatletter
\def\maxwidth{\ifdim\Gin@nat@width>\linewidth\linewidth\else\Gin@nat@width\fi}
\def\maxheight{\ifdim\Gin@nat@height>\textheight\textheight\else\Gin@nat@height\fi}
\makeatother
% Scale images if necessary, so that they will not overflow the page
% margins by default, and it is still possible to overwrite the defaults
% using explicit options in \includegraphics[width, height, ...]{}
\setkeys{Gin}{width=\maxwidth,height=\maxheight,keepaspectratio}
\IfFileExists{parskip.sty}{%
\usepackage{parskip}
}{% else
\setlength{\parindent}{0pt}
\setlength{\parskip}{1pt plus 1pt minus 1pt}
}
\setlength{\emergencystretch}{3em}  % prevent overfull lines
\providecommand{\tightlist}{%
  \setlength{\itemsep}{0pt}\setlength{\parskip}{0pt}}
\setcounter{secnumdepth}{0}
% Redefines (sub)paragraphs to behave more like sections
\ifx\paragraph\undefined\else
\let\oldparagraph\paragraph
\renewcommand{\paragraph}[1]{\oldparagraph{#1}\mbox{}}
\fi
\ifx\subparagraph\undefined\else
\let\oldsubparagraph\subparagraph
\renewcommand{\subparagraph}[1]{\oldsubparagraph{#1}\mbox{}}
\fi

% set default figure placement to htbp
\makeatletter
\def\fps@figure{htbp}
\makeatother


\date{}

\begin{document}
\begin{multicols}{3}
\subparagraph{created by wth}\label{header-c7}
\subparagraph{cause}\label{header-c7}

Truncation:计算量有关

Round-off:bit数相关


\subparagraph{不动点法}\label{header-c23}

收敛的充分条件:{[}a,b{]}内所有x满足\textbar{}g'(x)\textbar{}
\textless{} k, (0 \textless{} k \textless{} 1)

收敛速度: \( |p_{n+1} - p_n| <= k * |p_n - p_{n-1}|\) , k越小越快

\subparagraph{牛顿法}\label{header-c30}

用泰勒展开的前两项作为函数近似,求该近似函数的零点,在零点处再求函数近似,迭代。

\(p \approx p_0 - \frac{f(p_0)}{f'(p_0)}\)

牛顿法的k在p点处是0,所以收敛非常快

\subparagraph{迭代法误差分析}\label{header-c38}

\(lim \frac{p_{n+1} - p}{|p_n - p|^a} = \lambda\)

pn收敛于p, a越大,收敛速度越快

a = 1, linearly convergent

a = 2, quadratically convergent

\( g'(p) \neq 0\) 则最少是 linear convergent,拉格朗日中值定理可退

\( g'(p) = 0\) 时(如牛顿法) g不等于0的最高阶导数阶数a

牛顿法是二阶收敛的

\subparagraph{quadratica newtom method}\label{header-c54}

有重根(几重根都可以)的时候令 \( \mu(x) = \frac{f(x)}{f'(x)} \)

然后再做牛顿法 \( g(x) = x - \frac{\mu(x)}{\mu'(x)} \)

\subparagraph{Aitken's Method (Steffensen's Method)}\label{header-c61}

原理

\(p_{n+2} = p_n - \frac{( \Delta p_n )^2}{ \Delta ^ 2 p_n }\)

\( p_1 = g(p_0),\ p2=g(p_1)\)

\(p=p_0-\frac{(p_1-p_0)^2}{(p_2-2*p_1+p_0)}\)

\(p_0=p\)



\subparagraph{LU 分解}\label{header-c84}

如果矩阵不需要行交换就能高斯消元成上三角矩阵则可以LU分解(中间有0的话LU分解会没法进行)

复杂度\( n^3 / 3\)

\begin{enumerate}
\def\labelenumi{\arabic{enumi}.}
\item
  L对角线为1, U对角线为原始元素
\item
  \begin{enumerate}
  \def\labelenumii{\arabic{enumii}.}
  \item
    算U的第i行
  \item
    算L的第i列
  \item
    goto 1
  \end{enumerate}
\end{enumerate}

\subparagraph{Pivoting Strategies}\label{header-c104}

每次找一列中最大的元素,把最大元素所在的行交换到当前行

\subparagraph{Complete Pivoting}\label{header-c109}

到第i行时的时候在i右下角的矩阵里找最大的元素,然后通过交换行、交换列,换到\(a_{ii}\)

\subparagraph{Strictly Diagonally Dominant Matrix}\label{header-c114}

对角线上的元素绝对值严格大于此行其他元素绝对值之和

\subparagraph{Positive Define Matrix}\label{header-c114}

正定矩阵所有顺序主子式(左上角的前k行前k列构成的矩阵)的行列式>0

\subparagraph{LU factorization of a positive define
Matrix}\label{header-c119}

正定矩阵可以分解成\( L * L^t\)
的形式,L是下三角的L加上对角线替换为根号对角线

\subparagraph{Crout Reduction of Tridiagonal LInear
System}\label{header-c124}

\begin{enumerate}
\def\labelenumi{\arabic{enumi}.}
\item
  先LU分解, \(LUx = f\) 分步求解, 设\(y = Ux\)
\item
  \(Ly = f\) 
\item
  \(Ux = y \)
\end{enumerate}


\subparagraph{Norms 范数:}\label{header-c139}

定义:满足三个条件

\begin{enumerate}
\def\labelenumi{\arabic{enumi}.}
\item
  正定性 
\item
  同质性
\item
  满足三角不等式
\end{enumerate}

在某范数下收敛于x即与x的差的范数一致小于\(\epsilon\)

实空间里所有范数等价(在任意范数下收敛则所有范数下收敛)

\subparagraph{Vector Norm}\label{header-c160}

一阶范数:绝对值之和

二阶范数:欧拉距离

无穷范数:最大的元素绝对值

负无穷范数:最小的元素绝对值

\subparagraph{Matrix Norm}\label{header-c171}

定义比vector norm多了一项\\
4. consistence:\(||A B|| \leq ||A||\dot||B|| \)

一般有两种

Frobenius Norm

所有元素绝对值的平方和

Natural Norm

由向量范数导出,\(||A||_p=max_{||x||_p=1} ||Ax||_p\)

无穷范数:元素绝对值每一行的和的最大值

第一范数:元素绝对值每一列的和的最大值

第二范数:\(\sqrt{\lambda_{max}(A^TA)}\)
即\(A^TA\)这个矩阵最大的特征值的平方根,也就是谱半径,对于方阵来说就是特征值绝对值的最大值

\subparagraph{Spectral Radius}\label{header-c188}

\(\rho(A) = max|\lambda| \leq ||A||\)

谱半径等于特征值模长(可能是复数)的最大值,小于等于列元素绝对值和的最大值

A对称的时候即为第二自然范数

\(|\lambda|\cdot||x|| = ||\lambda x|| = ||Ax|| \leq ||A||\cdot ||x||\)

\subparagraph{Jacobi Iterative Method}\label{header-c199}

\(Ax = b\)

把A分为D, -L, -U三个矩阵相加

\((D-L-U)x = b\\
Dx = (L +U)x + b \\
x = D^{-1}(L + U) x + D^{-1}b \\
T = D^{-1}(L + U) \\
C = D^{-1}b\\
x = Tx + c\\\)

具体计算:

\(x_i = \frac{b_i - \sum_{j=1,j\neq i}^n{(a_{ij}X0_j)}}{a_{ii}}\)

可以做并行计算

\subparagraph{Gauss - Seidel Iterative Method}\label{header-c210}

\((D-L)x = Ux+b\)

不储存X0,每次直接用更新完的计算\(X_{i+1}\)

不能并行

\subparagraph{SOR(Successive Over Relaxation)}\label{header-c216}

\((D/\omega-L)X = ((1/\omega-1)D +U)X+b\)\\
\(x_i=x_i-\omega\frac{r_i}{a_ii},\ r_i=b_i-\sigma a_{ij}x_j \)\\
\(\omega>1\)X不保存,边算边用\\
\(\omega=1\)时即为Gauss-Seidel,\(\omega<1\)时为Under-Relaxation\\
kahan定理,只有\(0<\omega<2\)时,SOR才能收敛\\
Ostrowski-Reich定理,如果A正定而且\(0<\omega<2\),SOR对于任意初值收敛\\
\subparagraph{Convergency of Iterative Methods}\label{header-c220}

: The following statements are equivalent:

(1) A is a convergent matrix;

(2) \( lim_{n\to\infty} ||A^n|| = 0\) for some natural norm;

(3) \( lim_{n\to\infty}||A^n|| = 0\) for all natural norms;

(4) \( \rho(A) < 1\); 常用

(5) \( lim_{n\to\infty} A^nx = 0\)

\subparagraph{error bounds:}\label{header-c235}

\(||x - x_k|| \approx \rho (T)^k ||x - x_0||\)

T是Jaccobi的T

\subparagraph{Relaxation Methods}\label{header-c242}

\(X_ i^k = X_i^{k-1} - \omega(\frac{r_i^k}{a_{ii}})\)

\(r_i^k = b_i - \sum_{j<i}{a_{ij}x_j^k} - \sum_{j\ge i}{a_{ij}x_j^{k-1}} \)

\(\omega = 1\)时即为 Gauss - Seidel Iterative Method

\(T = I + \omega A\)


\subparagraph{拉格朗日基:}\label{header-c254}

第i个基在第i个插值点为1,其他插值点为0

\(L_{}n,i(x)= \prod_{j = 0, j \neq i}^n \frac{x - x_j}{x_i -x_j} \)\\

n是次数,从0开始

\(P_n(x) = \sum {L_{n,i}(x) y_i}\)

\subparagraph{Rolle's Theorem:}\label{header-c265}

n个零点所在的区间里必有一个点的n-1阶导数为0

\subparagraph{Remainder}\label{header-c270}

\( R(x) = f(x) - P_n(x) \)

\( g(t) = R(t) - K(x) \prod (t - x_i) \) 这个\(x\)是不等于\(x_i\)的任意固定值

根据Rolle's Theorem存在一个\(\zeta_x\)满足\(g^{(n+1)}(\zeta_x)=0\)
,带入上述两式,又因为\(P^{(n+1)}(\zeta_x)= 0\),推出

\(R_n(x) = \frac{f^{(n+1)(\zeta_x)}}{(n+1)!}\prod_{i=0}^n(x - x_i)\)

但是\(\zeta_x\)不一定能求得,常用 \(f^{(n+1)}(\zeta_x)\)
的一个上界来估算 \(R_n(x)\)


\subparagraph{Condition number}\label{header-c284}

\(||A||\cdot ||A^{-1}||\) is the key factor of error amplification, and
is called the condition number K(A). K(A)越大越难获得精确解

\(A(x+\delta x) = b + \delta b\)

\(\frac{||\delta x||}{||x||} \leq K(A) \cdot \frac{||\delta b||}{||b||} \)

\((A+\delta A)(x+\delta x) = b + \delta b\)

\(\frac{||\delta x||}{||x||} \leq \frac{K(A)}{1-K(A) \frac{||\delta A||}{||A||}} \cdot (\frac{||\delta A||}{||A||} + \frac{||\delta b||}{||b||}) \)

\subparagraph{Refinement}\label{header-c304}

\begin{enumerate}
\def\labelenumi{\arabic{enumi}.}
\item
  \(Ax = b\)
\item
  \(r = b - Ax\)
\item
  \(Ad = r\)
\item
  \(x = x + d\)
\end{enumerate}


\subparagraph{Power method}\label{header-c320}

\(x^k = Ax^{k-1}\)

相当于把一个随机向量塞到面团里,然后拉拉面,最后这个向量会跟拉面平行,即最大特征值对应的特征向量方向

要求最大特征值唯一,不能互为相反数(但可以两个完全相同)

\(\lambda \approx \frac{x_i^k}{x_i^{k-1}}\)

\subparagraph{Normalization}\label{header-c329}

\begin{enumerate}
\def\labelenumi{\arabic{enumi}.}
\item
  \( u^{k-1} = \frac{x^{k-1}}{|x^k-1|}\)
\item
  \(x^{k} = Au^{k-1}\)
\item
  \(\lambda = max(x^k_i)\)
\end{enumerate}

\subparagraph{Rate of Convergence}\label{header-c341}

收敛速率是\(|\lambda 2 / \lambda 1|\),更快的收敛速率要尽量让\(\lambda 2\)小,调整原点位置到\((\lambda 2 + \lambda n) / 2\)
可以再不产生新的\(\lambda2\)的情况下让\(\lambda 2 \)最小

\subparagraph{Inverse Power Method}\label{header-c346}

可以求得绝对值最小的特征值

求\(p_0\)附近的特征值:

\begin{enumerate}
\def\labelenumi{\arabic{enumi}.}
\item
  \(B= A- p_0  I\) 
\item
  \(x^k = B^{-1}x^{k-1}\)
\item
  \(\frac 1 \lambda = x^k / x^{k-1}\)
\end{enumerate}

但是导致condition number降低


一般内插比外插要准确

拉格朗日插值如果新增加一个插值点需要全部重算

下面两种方法都更方便于添加插值点

\subparagraph{Neville's Method:}\label{header-c373}

用两个同阶的p合并可以得到更高阶的p

\(p_{1,2,3,4}(x) = \frac{(x -x_4)p_{1,2,3} - (x - x_1)p_{2,3,4}}{x_1 - x_4}\)

\subparagraph{Newton's Interpolation}\label{header-c380}

\( f[x_0 \cdots x_k] =\frac{ f[x_ 0 \cdots x_{k-1}] - f[x_1 \cdots x_k] }{x_0-x_k}\)

\( f(x) = f[x_0] + f[x_0,x_1] (x-x_0)+f[x_ 0 ,\cdots, x_n](x-x_0) \cdots (x - x_n)\)

\subparagraph{Hermite Interpolation}\label{header-c391}

插值满足在n个点出给出的值和m个点处给出的斜率

\( H(x) = \sum_{i=0}^n f(x_i)h_i(x) + \sum_{i=0}^{m}f'(x_i)\hat h_i(x)\)

where \(h_i(x_i) = (i == j)\), \(h'_i(x_j) = 0\), \(\hat h_i(x_j) = 0\),
\(\hat h'_j(x_j) = (i == j) \)

推出

\(h_i(x)=[1-2L'_{n,i}(x_i)(x-x_i)]L^2_{n,i}(x)\\
\hat h(x)=(x-x_i)L_{n,i}^2(x)\)

\subparagraph{Cubic Spline Interpolation}\label{header-c402}

解决了随着插值点的增多插值函数并不收敛于原函数的问题

方法是用子区间三阶插值来拟合原函数


\subparagraph{Least Squares Approximation}\label{header-c408}

m个点,用n阶多项式函数P你和f,\(n \ll m\)

使得\(E = \sum_1^m[P_n(x_i) - y_i]^2\)最小

即求P的系数\(a\)满足E最小,求偏导可得

Let \( b_k = \sum_1^mx_i^k\),\( c_k=\sum_1^m{y_ix_i^k}\)

出题的话一般求导令导数等于0即可,计算机则解下列矩阵
$
\begin{pmatrix}
b_{0+0}  & \cdots & b_{0+n} \\
\vdots & \ddots & \vdots \\
b_{n+0} & \cdots & b_{n+n} 
\end{pmatrix}
\begin{pmatrix}
a_0\\ \vdots \\ a_n
\end{pmatrix}
=
\begin{pmatrix}
c_0\\ \vdots \\ c_n
\end{pmatrix}
$
P是非线性函数的时候使用换元法变成线性,直接求导应该也是可以的

\subparagraph{General Least Squares Approximation}\label{header-c422}

\(E = \sum _1^mw_i[P_n(x_i) - y_i]^2\)

\(E = \int _1^mw(x)[P(x) - f(x)]^2\)

如果\((f, g)\)则两个基正交

\((f,g) =\begin{cases} 
\sum_1^m w_i f(x_i)g(x_i) \\
\int_a^bw(x)f(x)g(x)dx
\end{cases}\)

设\(P(x)=a_0\varphi_0(x)+\ldots+a_n\varphi_n(x)\),求\(a\)
$
\begin{pmatrix}
 \\
b_{ij}=(\varphi_i, \varphi_j)\\
\\
\end{pmatrix}
\begin{pmatrix}
a_0\\ \vdots \\ a_n
\end{pmatrix}
=
\begin{pmatrix}
(\varphi_0, f)\\ \vdots \\ (\varphi_0, f)
\end{pmatrix}
$
取\(\varphi_j(x)=x^j\)时,\((\varphi_i, \varphi_j)=\frac{1}{i+j+1}\),不是正交的,Hilbert
Matrix,不好算,条件数大

\subparagraph{构造正交基}\label{header-c434}

\(\varphi_0(x) = 1 \\
\varphi_k(x) = (x-B_k)\varphi_{k-1}(x)-C_k\varphi_{k-2}(x)\\
B_k=\frac{(x\varphi_{k-1}, \varphi_{k-1})}{(\varphi_{k-1}, \varphi_{k-1})}\\
C_k=\frac{(x\varphi_{k-1}, \varphi_{k-2})}{(\varphi_{k-2}, \varphi_{k-2})}\\\)


\subparagraph{forward backward}\label{header-c437}

拉格朗日插值分析,误差级别是O(h)

\subparagraph{积分近似}\label{header-c440}

用拉格朗日插值的积分近似

\(\int_a^bf(x)dx = \sum_0^nf(x_k)\int_a^bL_k(x)dx\)

对于等间距取点的情况,后面朗格朗日项的积分不依赖于f或者区间,所以说是stable的

\subparagraph{误差估计}\label{header-c446}

使用拉格朗日余项积分获得误差

\(R[f] = \int_a^b\frac{f^{(n+1)(\zeta_x)}}{(n+1)!}\prod_{i=0}^n(x - x_i)\)

\subparagraph{积分准确度}\label{header-c450}

对几阶多项式是完全准确的

\subparagraph{Simpson's Rule}\label{header-c455}

\(\int_a^bf(x)dx = \frac{b-a}{6}[f(a) + 4f((a+b)/2)+f(b)]\)


\subparagraph{Chebyshev Polynomial}\label{header-c458}

Chebyshev Polynomial如下,两种表达方式

\(T_0(x)=1,\ T_1(x)=x\\
T_{n+1}(x)=2xT_n(x)-T_{n-1}(x)\\
T_n(x)=cos(n\ arccos(x))\)

该多项式是用于选择取样点的,用该多项式选取的取样点插值可以使得无穷范数最小,即最大的误差最小。最大误差处的点称为deviation
point。

原理:余项中的\(w_n(x)\)最小(余项中仅有此项与点选取有关)

\(w_n(x)=\prod(x-x_i)\\
w_n(x)=x^n-P_{n-1}(x)\\
由切比雪夫定理知,P_{n-1}有n+1个deviation\ points.\)

目标为寻找一个\(P_{n-1}(x)\)使得\(w_n(x)\)最小,用下式取点

\(T_n(x)\ has\ n\ root\ 则选取点\\x_k=cos(\frac{2k-1}{2n}\pi)\)

确定点后用拉格朗日插值

\subparagraph{Economization}\label{header-c473}

减去高阶切比雪夫多项式达到降阶的目的

n阶mono chebyshev多项式(最高项系数归一化)的最大值\(\frac{1}{2^{n-1}}\)

\subsubsection{Chatper 4}\label{header-c478}

\subparagraph{Composite Numerical Integration}\label{header-c479}

分段过细拟合会导致过拟合(采样点太多,阶数变高)

所以要分段拟合积分

\subparagraph{Composite Trapezoidal Rule}\label{header-c484}

\(\int_a^bf(x)dx\approx\frac h 2 [f(a) + 2\sum_{k=1}^{n-1}f(x_k)+f(b)] = T_n\\
R[f] = -\frac {h^2} {12}(b-a)f''(\xi_)\)

\subparagraph{Composite Simpson`s Rule}\label{header-c486}

\(h=\frac{2(b-a)}{n}\\
\int_{x_k}^{x_{k+2}}f(x)dx\approx\frac h 6 [f(x_k) + 4f(x_{k+1})+f(x_{k+2})]\\
\int_a^bf(x)dx \approx \frac h 3 [f(x_0) + 2\sum^{n/2-1}_{j=1}f(x_{2j})+4\sum^{n/2}_{j=1}f(x_{2j-1})+f(x_n)] + R[f]\\
R[f] = -\frac{b-a}{180}h^4f^{(4)}(\xi)\)

\subparagraph{Romberg Integration}\label{header-c5}

迭代直到\(S_{k,0}\)与\(S_{k-1,0}\)误差小于\(\epsilon\)

\(S_{0,n}=T_n\\
S_{k,n}=\frac{4^kS_{2n}-S_n}{4^k-1}\)

计算顺序如\(T_1, T_2, T_4,T_8,S_1,S_2,S_4,C_1,C_2,R_1\)

余项\(R_{2n}[f] = -(\frac h 2)^2\frac{1}{12}(b-a)f''(\xi_)\approx\frac 1 4 R_n[f]\)

\subparagraph{Richardson's Extrapolation}\label{header-c51}

某种插值方法\(T_0\)的误差为\(T_0-I=a_1h+a_2h^2+a_3h^3+\cdots\),则可使用下式迭代减小误差

\(T_m(h)=\frac{2^m T_{m-1}(\frac h 2) - T_{m-1}(h)}{2^m-1}=I+\delta_1 h^{m+1}+\delta_2 h^{m+2}+\cdots\)

\subparagraph{Adaptive Quadrature Methods}\label{header-c61}

自动在变化剧烈的地方多采样

simpson积分,分为两个区间后得到两个等式,假设等式中的导数相等,得到一个error关于整个区间的辛普森和两个半区间的辛普森的表达式。当error小于给定值时即可停止区间细分,否则继续细分

\(\int_a^bf(x)dx=S(a,b)+\frac{h^5}{90}f^{(4)}(\xi_1)\\
\int_a^bf(x)dx=S(a,\frac{a+b}{2})+S(\frac{a+b}{2},b)+\frac {1}{16}\frac{h^5}{90}f^{(4)}(\xi_2)\\
Assuming\ f^{(4)}(\xi_1)\approx f^{(4)}(\xi_2)\\
\epsilon=\frac {1}{15}|S(a,b) - S(a ,\frac{a+b}{2}) -S(\frac{a+b}2,b)|\)

\subparagraph{Gaussian Quadrature}\label{header-c75}

目标:用尽量少的点得到2n+1阶的精确积分拟合(在w(x)意义下的积分),注意是积分,而不是函数,只能得到一个精确的值而不是函数表达式。即

\(\int w(x) f(x)=\sum A_kf(x_k)\)

高斯积分只用n+1个点的采样即可做到。

如何得到这n个点:

以这n+1个点为根的多项式\( W(x)=\prod_0^n(x-x_i) \),满足与所有小于等于n阶的多项式正交。利用正交化可解出\(W(x)\)

证明:


精确=\textgreater{}正交

\(q \leq2n+1,\ m \leq n\\
\int w(x)Q_q(x)=\int w(x)P_m(x)W(x)=\sum A_kP_m(x)W(x_k) = 0\)

正交=\textgreater{}精确

\(Q(x)=W(x)Q(x)+r(x)\\
\int w(x)Q(x)=\int w(x)W(x)q(x) + \int w(x)r(x)=0+\int w(x)r(x)\\
=\sum_0^nA_kr(x_k)=\sum_0^n[A_kW(x_k)q(x_k)+A_kr(x_k)]=\sum_0^nA_kQ(x_k)\)



\subparagraph{Euler method}\label{header-c99}

\(w_{i+1}=w_i+h f(t_i, w_i)\)

误差\(|y_i-w_i|\leq\frac{hM}{2L}[e^{L(ti-a)}-1]\)

yi是精确值,wi是数值值,M是\(max(|y''(x)|)\),L是lipschitz常数,ti是第i个点x,a是起点

有round off error的时候,\(\delta\)是误差的界

误差\( |y_i-w_i|\leq\ \frac 1 L (\frac{hM}{2}+\frac \delta h)[e^{L(ti-a)}-1] + |\delta _0|e^{L(t_i-a)}\)

\subparagraph{truncation error}\label{header-c110}

\(\tau_{i+1}=\frac{y_{i+1}-y_i}{h}-f(t_i,y_i)=\frac h 2 y''(\xi)\)

\subparagraph{Implicit Euler's Method}\label{header-c113}

\(\tau_{i+1}=\frac{y_{i+1}-y_i}{h}-f(t_i,y_i)=-\frac h 2 y''(\xi)\)

\subparagraph{Trapezoidal Method ( modified Euler's Method
)}\label{header-c116}

\(w_{i+1}=w_i+\frac h 2[f(t_i, w_i) + f(t_{i+1}, w_{i+1})]\)

local truncation error \(O(h^2)\)

\subparagraph{Double-Step Method}\label{header-c121}

\(w_{i+1}=w_{i-1}+2hf(t_i,w_i)\)

local truncation error \(O(h^2)\)

\subparagraph{Runge-Kutta Method}\label{header-c126}

\(w_{i+1}=w_i+(\lambda _1 + \lambda _2) h y'(t_i) + \lambda _2ph^2y''(t_i) + O(h^3)\)

\(\lambda _1 + \lambda _2=1,  \lambda _2 p=\frac 1 2\)

local truncation error \(O(h^2)\)

\(w_{i+1}=w_i+\frac h 6(K_1+2K_2+2K_3+K4)\\
K_1=f(t_i,w_i)\\
K_2=f(t_i+\frac h 2,w_i+\frac h 2 K_1)\\
K_3=f(t_i+\frac h 2,w_i+\frac h 2 K_2)\\
K_4=f(t_i+h,w_i+h K_3)\\\)

\subparagraph{Multi-Step Method}\label{header-c121}

\(w_{i+1}=a_1 w_i + a_2 w_{i-1} + h(b_1 f_i + b_2 f_{i-1})\)
求\(a_1, a_2, b_1, b_2\),
使用泰勒展开,\(w_j\)全部展开成从\(w_i\)出发,\(f_j\)全部展开成从\(f_i\)出发。
展开后带入原式,比较系数


\subparagraph{Stability}\label{header-c135}

假设某种迭代方法如下

\(w_{i+1}=a_{m-1}w_i+a_{m-2}w_{i-1}+\cdots+a_0w_{i+1-m}+hF(t_i,h,w_{i+1},w_i,\cdots,w_{i+1-m})\)

则该积分的特征多项式\(P(\lambda)=\lambda^m-a_{m-1}\lambda^{m-1}-\cdots-a_1\lambda-a_0=0\)

解得特征根,如果所有\(|\lambda_i|\leq1\),并且模长为1的根都是单根,则称为满足root
condition

\begin{enumerate}
\def\labelenumi{\arabic{enumi}.}
\item
  满足root condition并且模长为1的特征根只有1的方法,称为strongly stable
\item
  满足root condition并且有多个模长为1的特征根的方法,称为weakly stable
\item
  不满足root condition的方法称为unstable
\end{enumerate}
\end{multicols}
\end{document}
